\section{Berechnung der Messunsicherheiten}
Für die Unsicherheit der ATP Stoffmengen wurde eine relative Unsicherheit
von 5\% durch Pippetierfehler angenommen. $N_{err} = 5 \%$
Bei der Ausmessung Aktingeschwindigkeit wurden mindestens 10 Aktinfasern gewählt,
die sich im Zeitintervall von 10 Sekunden durchgehend bewegt haben.
Aus den gemessenen Distanzen wurde der Mittelwert, und die Varianz bestimmt.
\[
  \text{Pixel}_{err} = \sqrt{\sigma} + 2
\]
Zu dem Distanzfehler noch der Auflösungsfehler des Mikroskops
von $2 \cdot \text{Pixellänge}$ aufaddiert. Der Faktor 2 kommt daher,
dass man am Anfang und am Ende der Aktinfaser den Fehler über eine Pixellänge hat.
\[
  v_{err} = \frac{\text{Pixel}_{err} \cdot \text{Pixellänge}}{10s}
\]
Die Pixellänge wurde uns mit $1.57\sfrac{\mu m}{\text{Pixel}}$ angegeben.
Der Zeitfehler wurde vernachlässigt.
Der lineare Fit nach Formel \ref{equ:michaelis_inverse} gibt die Fitparameter
$m = \sfrac{K_m}{v_{max}} \pm m_{err}$ und $t = \sfrac{1}{v_{max}} \pm t_{err}$ an.
\begin{equation}
	\sigma_{f(x_1,x_2,...)} = \sqrt{\sum\limits_{i=1}^n (\frac{\partial f}{\partial x_i} \cdot \sigma_{x_i})^2}
	\label{equ:staterr}
\end{equation}
Mit $K_m = \sfrac{m}{t}$ und $v_{max} = \sfrac{1}{t}$ und unter Verwendung der
gausschen Fehlerfortpflanzung \ref{equ:staterr} wurden die
Fehler des Fits mit
\[
  K_{m\ err} = \sqrt{(\frac{m_{err}}{t})^2 + (\frac{m \cdot t_{err}}{t^2})^2}
\] 
\[
  v_{max\ err} = \frac{t_{err}}{t^2}
\]
errechnet.
Ensprechend wurden die Fehlerwerte für die Fehlerbalken Berechnet.
Für die Berechnungen und das Plotten der Graphen wurde ein Pythonskript geschrieben.
Dieses kann auf der Webseite \url{https://github.com/JackTheEngineer/molecular_motors}
im Ordner \texttt{calc/} eingesehen werden.

%%% Local Variables:
%%% mode: latex
%%% TeX-master: "../motors.tex"
%%% End:
